\documentclass[10pt,oneside,a4paper]{article}

\usepackage[latin1]{inputenc} 
\usepackage[italian]{babel}
\usepackage{siunitx} %Inserisce automaticamente i dati con le unit� di misura correttamente formattate del SI (utilizzo: \SI{0.82}{m^2}, in generale \SI{misura con il punto decimale}{unit� di misura})
\usepackage{listings} %Per citare codice informatico formattandolo correttamente
\usepackage{amsmath}
\usepackage{graphicx}
\usepackage{epigraph}

\setcounter{section}{-1}

\title{\textsc{Misura della \emph{densit�} e altre cose belle.}}
\author{\small{G. Galbato Muscio} \and \small{L. Gravina} \and \small{L. Graziotto} \and \small{M. Rescigno}}
\date{}

\begin{document}
\begin{figure}
	\centering
	\includegraphics[scale=0.5,trim={2.8cm 8.9cm 0 9cm},clip]{logo.png}
\end{figure}
\maketitle
\begin{center} 
\fbox{{\fontsize{13pt}{8mm}\textsc{Gruppo C2.3}}} \\
\vspace{1cm}
\begin{tabular}{ccc}
	 Esperienza di laboratorio && Consegna della relazione \\
	  \emph{\small{30 febbraio 2017}} && \emph{\small{\today}} \\
\end{tabular} 

\vspace{0.5cm}

\end{center}
\hrule
\vspace{0.5cm}
\begin{abstract}
\[
	d = m/V \qquad [d] = \SI{}{Kg/dm^3}
\]
La densit� � bellissima e poco conosciuta per cui cercheremo qui e ora di descriverne un metodo di determinazione olistico.
\end{abstract}
\vspace{0.5cm}
\tableofcontents %Indice
\pagebreak
\section{Convenzioni}
In questa relazione verranno usate le seguenti convenzioni:
\begin{enumerate}
	\item sar� usata la virgola [ $,$ ] come separatore decimale;
	\item l'approssimazione decimale della cifra $5$ sar� fatta controllando, ove possibile, la cifra successiva (in ordine di lettura) al 5, in particolare se la cifra � compresa tra $0$ e $4$ (compresi) l'arrotondamento avverr� per difetto (es. $0,153 \rightarrow 0,1$), se la cifra � invece compresa tra $5$ e $9$ l'arrotondamento sar� per eccesso (es. $0,156 \rightarrow 0,2$), se non � possibile controllare la cifra successiva al 5 l'arrotondamento sar� fatto per difetto.
\end{enumerate}


\section{Scopo e descrizione dell'esperienza}

\end{document}