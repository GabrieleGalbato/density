\documentclass[10pt,oneside,a4paper]{article}

\usepackage[latin1]{inputenc} 
\usepackage[italian]{babel}
\usepackage{siunitx} %Inserisce automaticamente i dati con le unit� di misura correttamente formattate del SI (utilizzo: \SI{0.82}{m^2}, in generale \SI{misura con il punto decimale}{unit� di misura})
\usepackage{listings} %Per citare codice informatico formattandolo correttamente
\usepackage{amsmath}
\usepackage{graphicx}
\usepackage{epigraph}

\setcounter{section}{-1}

\title{\textsc{Misura della \emph{densit�}.}}
\author{\small{G. Galbato Muscio} \and \small{L. Gravina} \and \small{L. Graziotto} \and \small{M. Rescigno}}
\date{}

\begin{document}
\begin{figure}
	\centering
	\includegraphics[scale=0.5,trim={2.8cm 8.9cm 0 9cm},clip]{logo.png}
\end{figure}
\maketitle
\begin{center} 
\fbox{{\fontsize{13pt}{8mm}\textsc{Gruppo B2.3}}} \\
\vspace{1cm}
\begin{tabular}{ccc}
	 Esperienza di laboratorio && Consegna della relazione \\
	  \emph{\small{27 marzo 2017}} && \emph{\small{3 aprile 2017}} \\
\end{tabular} 

\vspace{0.5cm}

\end{center}
\hrule
\vspace{0.5cm}
\begin{abstract}
\[
	\rho = \frac{m}{V} \qquad [\rho] = \SI{}{\frac{kg}{m^3}}
\]
La densit� � bellissima e poco conosciuta per cui cercheremo qui e ora di descriverne un metodo di determinazione olistico.
\end{abstract}
\vspace{0.5cm}
\tableofcontents %Indice
\pagebreak
\section{Convenzioni}
Qui introdurremo le convenzioni usate. Sappiate che per quanto riguarda l'arrotondamento useremo la mia (Gabriele) convenzione personale in quanto sono arrivato per primo: arrotonderemo $x,5$ in base a ci� che verr� dopo il $5$, cio� per esempio 
\[
	4,351 \rightarrow 4,3 \qquad 4,356 \rightarrow 4,4.
\]
Consiglio inoltre di utilizzare la \emph{comma} (",") come separatore decimale in luogo del punto per almeno due motivi:
\begin{enumerate}
	\item siamo italiani e dobbiamo preservare le nostre tradizioni invece di adattarci a quelle degli aglosassoni (per ricordarvelo, quelli l� usano i \emph{galloni} per misurare i volumi),
	\item il pacchetto \emph{siunitx} che gestisce le unit� di misura secondo il SI utilizza la comma come separatore decimale.
\end{enumerate}

Questa � una prova per vedere se compila una delle tabelle caricate da Luca.
\begin{table}[ht]
\centering
\begin{tabular}{rrrrr}
  \hline
 & Altezza-Palmer(mm) & Altezza-nonio(mm) & Diametro-palmer(mm) & Massa(g) \\ 
  \hline
Error: & 0.0010 & 0.0500 & 0.0010 & 0.0003 \\ 
  Mean: & 20.0165 & 20.0867 & 11.9827 & 6.1080 \\ 
  1 & 20.0180 & 20.1500 & 11.9750 & 6.1070 \\ 
  2 & 20.0210 & 20.0500 & 11.9850 & 6.1080 \\ 
  3 & 20.0190 & 20.1000 & 12.0180 & 6.1070 \\ 
  4 & 20.0200 & 20.1000 & 11.9740 & 6.1070 \\ 
  5 & 20.0190 & 20.1000 & 11.9830 & 6.1080 \\ 
  6 & 20.0110 & 20.0500 & 11.9850 & 6.1080 \\ 
  7 & 20.0210 & 20.1000 & 11.9780 & 6.1050 \\ 
  8 & 20.0190 & 20.1000 & 11.9790 & 6.1080 \\ 
  9 & 20.0150 & 20.1000 & 11.9860 & 6.1070 \\ 
  10 & 20.0110 & 20.1500 & 11.9800 & 6.1090 \\ 
  11 & 20.0110 & 20.1000 & 11.9820 & 6.1090 \\ 
  12 & 20.0150 & 20.0500 & 11.9790 & 6.1090 \\ 
  13 & 20.0110 & 20.0500 & 11.9890 & 6.1080 \\ 
  14 & 20.0150 & 20.1000 & 11.9900 & 6.1090 \\ 
  15 & 20.0200 & 20.0500 & 11.9810 & 6.1090 \\ 
  16 & 20.0120 & 20.0500 & 11.9820 & 6.1080 \\ 
  17 & 20.0090 & 20.0500 & 11.9780 & 6.1070 \\ 
  18 & 20.0150 & 20.1000 & 11.9840 & 6.1090 \\ 
  19 & 20.0150 & 20.1000 & 11.9800 & 6.1090 \\ 
  20 & 20.0190 & 20.0500 & 11.9710 & 6.1080 \\ 
  21 & 20.0250 & 20.0500 & 11.9890 & 6.1090 \\ 
  22 & 20.0190 & 20.0500 & 11.9840 & 6.1080 \\ 
  23 & 20.0200 & 20.1000 & 11.9800 & 6.1070 \\ 
  24 & 20.0100 & 20.1000 & 11.9910 & 6.1080 \\ 
  25 & 20.0100 & 20.1000 & 11.9800 & 6.1070 \\ 
  26 & 20.0210 & 20.0500 & 11.9750 & 6.1080 \\ 
  27 & 20.0200 & 20.1000 & 11.9920 & 6.1090 \\ 
  28 & 20.0250 & 20.1000 & 11.9810 & 6.1100 \\ 
  29 & 20.0100 & 20.1000 & 11.9730 & 6.1080 \\ 
  30 & 20.0200 & 20.1500 & 11.9780 & 6.1080 \\ 
   \hline
\end{tabular}
\end{table}


\section{Scopo e descrizione dell'esperienza}

\end{document}